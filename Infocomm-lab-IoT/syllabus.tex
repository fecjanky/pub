\documentclass[a4paper]{article}

\usepackage[utf8]{inputenc}

\usepackage{url}
\usepackage[hidelinks]{hyperref}

\usepackage{caption}

\usepackage{listings}

\usepackage{color}

\usepackage{enumitem}

% *** GRAPHICS RELATED PACKAGES ***
%\usepackage[pdftex]{graphicx}
\usepackage{graphicx}
%\usepackage[dvips]{graphicx}
% to place figures on a fixed position
\usepackage{float}

\usepackage[margin=1in]{geometry}

\title{IoT Lab}
\author{}
\date{}


\begin{document}

\maketitle

\tableofcontents

\section{IoT lab measurement exercises}

In the first phase of the lab a micro-controller will be transformed into a mote. A sensor mote is capable of detect
various parameters of its environment and is capable of communication. For this purpose a sensor and a radio module is
going to be attached to the micro-controller. For this same device an actuator is going to be connected to it.
Furthermore a simple display device is also connected to the already built hierarchy.

In the second phase of the lab a virtual device is created using one of the cloud IoT provider's services that is going
to accept and process the data arriving from our physical sensor. The service is going to store and display the data.
Furthermore a virtual controller is attached to the virtual sensor that is going to send control signals for the
physical device. A gateway device is used for translating the data into the right format between the physical and
virtual devices.

The arrangement of these components can be seen on Figure~\ref{fig:meas-arrangement}. The task of the students
attending this lab is to program the devices and the gateway for the achieving the desired operation.

\begin{figure}[H]
    \centering
    \includegraphics[width=0.9\textwidth]{figures/devices-arrangement.png}
    \caption{Arrangement of devices and components used in this lab}
    \label{fig:meas-arrangement}
\end{figure}

During the exercises a lab report has to be created. In this report the students must document the actions carried out,
the code written, the configuration files created etc. The lab report can also contain screen captures. The basic
guideline is to create such a document of which the measurement is easily reproducible.

\section{Getting started with \emph{\textbf{mbed}} environment}

mbed is an IDE (and also an operating system) tailored for IoT applications based 32 bit ARM micro-controllers. There
are several commercially available boards available as a result it allows simple and rapid prototyping. One of its main
advantages compared to other IDEs that it can handle multiple types of micro-controllers so that the code written can
be reused in multiple environments. Furthermore its UI is intuitive and also supports on-line workflows with team
integration and version control. Naturally there are some cons as well namely that from mbed the low-level hardware
components are not reachable from it and sometimes the basis of the framework called \emph{mbed OS} might contain bugs.

The micro-controller used during this lab is of type NUCLEO-F446RE.

\subsection{First steps}

Before proceeding any further register on site \url{https://developer.mbed.org/}. The registration is simple and easy
and doesn't require instant verification of our e-mail addresses. After successful registration log-in on the website
and then
\begin{enumerate}
    \item Go to ``Platforms" tab where the development board is selected
    \item Filter the results for manufacturer of "STMicroelectronics"
\end{enumerate}
\begin{figure}[H]
    \centering
    \includegraphics[width=0.9\textwidth]{figures/mbed-platform.png}
\end{figure}
\begin{enumerate}[resume]
    \item Search for and select ``NUCLEO-F446RE" dev board. Here we can find detailed description on the micro-controller
          and also the schematics for the pin to feature connector assignment and functions associated with them
\end{enumerate}
\begin{figure}[H]
    \centering
    \includegraphics[width=0.9\textwidth]{figures/mbed-nucleo.png}
\end{figure}
\begin{enumerate}[resume]
    \item Add the selected board to the compiler by clicking on the ``Add to your mbed Compiler" and the open it using the
          link in the upper right corner (5.)
\end{enumerate}
\begin{figure}[H]
    \centering
    \includegraphics[width=0.9\textwidth]{figures/mbed-compiler.png}
\end{figure}

\subsection{Sample programs and binary upload to the device}

Click on ``New" in the top left corner (1.). In the pop-up window (2.) select the appropriate device type, and then we
can select from one of the existing sample projects (or alternatively create a new one) and give a name for it. With
the tick-box in the bottom we can select whether to update our code if one of the dependent libraries are updated.
Click on OK and afterwards the newly created program will be visible under the My Programs section with the chosen
name.
\begin{figure}[H]
    \centering
    \includegraphics[width=0.9\textwidth]{figures/mbed-new.png}
\end{figure}

We can open and edit (1.) the source file by clicking on it in the left pane. As soon as we are done with the
modifications the compilation can be started by clicking on ``Compile" button (2.). If there were no errors and the
program compiles successfully the result will be file download pop-up for the resulting binary (3.)

\begin{figure}[H]
    \centering
    \includegraphics[width=0.9\textwidth]{figures/mbed-compile.png}
\end{figure}

Download/save this binary on the volume created by the dev board (NUCLEO). The green LED will start blinking during
software upload to the micro-controller. If there are no errors our code will run on the device.

\begin{figure}[H]
    \centering
    \includegraphics[width=0.9\textwidth]{figures/mbed-nucleo-save.png}
\end{figure}

\subsection{Creating an empty project}
Similarly to the previous steps click on ``New" (1.) then select "Empty Program" option (2.). Name the project (3.)
then click OK. (4.)

\begin{figure}[H]
    \centering
    \includegraphics[width=0.9\textwidth]{figures/mbed-empty-proj.png}
\end{figure}

The empty application project needs at least the mbed runtime library that can be added to the project in the following
way:
\begin{enumerate}
    \setcounter{enumi}{0}
    \item Select the project
    \item click on ``Import"
    \item select the ``Libraries" tab
    \item search for the word ``mbed"
    \item Import
\end{enumerate}

\begin{figure}[H]
    \centering
    \includegraphics[width=0.9\textwidth]{figures/mbed-import.png}
\end{figure}

\subsection{Different ways of importing}

There are some other ways for importing programs and libraries. A program can be imported based on the above describe
method but selecting the ``Programs" tab (1.).
A program or library can be either uploaded from our local machine from a file (2.) or we can download it given that
the url of it is know and we have sufficient privileges.

\begin{figure}[H]
    \centering
    \includegraphics[width=0.9\textwidth]{figures/mbed-import2.png}
\end{figure}

In case if we are a part of a development team there is an option for importing the sources published by other
developers.

\begin{figure}[H]
    \centering
    \includegraphics[width=0.9\textwidth]{figures/mbed-online-import.png}
\end{figure}
\begin{figure}[H]
    \centering
    \includegraphics[width=0.9\textwidth]{figures/mbed-online-import2.png}
\end{figure}

\section{Attaching sensors and actuators}

Sensors are available as standalone modules (or simple components). After they are attached to the controller the
sensed data is sent to the micro-controller. The transmission of the data can vary. Simple sensors provide analogue
signals that is digitalized by the micro-controller's A/D converter. More complex sensors digitalize analogue signals
and transmit digital data towards the micro-controller. These devices provide configuration capabilities on top of
reading out data. The communication with these sensors are mainly consists of transmitting command words that is
executed by the sensor and returns the results of the operation for the micro-controller. Complex sensors are capable
of managing other sensors and return aggregated data for the micro-controller (sensor fusion). The physical
communication standard between a sensor and the micro-controller is usually one of SPI, I2C, USART, OneWire, CAN, etc.

The sensors used in this lab can be retrieved from the lab demonstrator. A wide variety of sensors are available
ranging from very simple to more complex ones. A complex sensor does not necessary implies complex attachment
procedure. If one can't find a readily available code for connecting a given sensor it can be written during the class.

\section{Connecting a communication module}

\subsection{Nucleo board pinout}

The Nucleo F446RE board's pinout is shown of Figure~\ref{fig:nucleo-pinout}.

\begin{figure}[H]
    \centering
    \includegraphics[width=0.9\textwidth]{figures/nucleo-pinout.png}
    \caption{Nucleo F446RE pinout}
    \label{fig:nucleo-pinout}
\end{figure}

The pins labeled are the female connectors that can be easily addressed from the mbed development environment.
Furthermore all of the 64 male connectors of the Nucleo board is accessible. The male connector header is found right
beside the female connector header the only twist is that in the right side female header there is a half pin pause due
to providing Arduino compatibility. The neighboring male header does not have this pause that means for a given female
connector it's corresponding male connector is located in the bottom right position.
Figure~\ref{fig:nucleo-pinout-morpho} illustrates the male connector pinout.

\begin{figure}[H]
    \centering
    \includegraphics[width=0.9\textwidth]{figures/nucleo-morpho-pinout.png}
    \caption{Nucleo F446RE morpho pinout}
    \label{fig:nucleo-pinout-morpho}
\end{figure}

When supplying power for external devices extreme caution must be taken about the components that require 3.3V power
supply \textbf{MUST NOT} be connected to a 5V supply. (e.g. the radio module requires 3.3V supply).
3.3V power supply is available in three locations: on the Arduino headers there is one male and one female connector
and also on the outer header the 3rd pin from the top on the left side (according to the image).

\subsection{NRF24L01+ module pinout}

The NRF24L01+ radio module is a device communicating on SPI bus. It's important that the power supply must not exceed
3.3V!
In case of improper connection the device will be damaged permanently.
This module has power supply pins (GND,VCC) and SPI bus pins (MISO,MOSI,SCK) and 2 SPI bus control pins (CE,CSN).
Furthermore there is an interrupt pin (IRQ) that is not used in these exercises so that it can be left not connected.
The pinout of the NRF24L01+ module's pinout is show of Figure~\ref{fig:radio-pinout}

\begin{figure}[H]
    \centering
    \includegraphics[width=0.9\textwidth]{figures/radio-pinout.png}
    \caption{NRF24L01+ pinout}
    \label{fig:radio-pinout}
\end{figure}

\subsection{Connecting Nucleo and NRF24L01+ modules}

Based on the pinout diagrams connecting the two devices is trivial. There is only one complication about CE and CSN
pins because they can be placed freely. It is advised that D9 and D10 pins are used for this purpose because in that
case they will be collocated with the SPI pins. It must not be forgotten that the radio software has to be adjusted
according to the actual pinout used (NodeConfig class). An example of the interconnection is shown of
Figure~\ref{fig:radio-nucleo-connection}
A sample program for the radio module connection can be downloaded from
\url{https://www.tmit.bme.hu/sites/default/files/attachments/nRF24Test_TMRh20_zip_nucleo_f446re.zip}.

\begin{figure}[H]
    \centering
    \includegraphics[width=0.9\textwidth]{figures/board-radio-example.png}
    \caption{An example of interconnecting Nucleo F446RE and NRF24L01+}
    \label{fig:radio-nucleo-connection}
\end{figure}

\section{Creating a \emph{virtual device} and communicating with it through DeviceHub.net}

There are multiple service providers offering storage space for IoT applications. It is not only possible to store the
data but also to analyze it. The data upload is realized using the Internet services for which the providers offer an
API. The most popular upload and access mechanism is based on the HTTP protocol but \emph{MQTT} protocol is also
utilized. For this measurement an MQTT based one is going to be used. MQTT is a publish/subscribe based protocol where
the data is sent over appropriate channels (topic) and symmetrically if we want to access some data we need to
subscribe on the desired channel.

\subsection{Creating a project}

As an initial step register on \url{https://dashboard.devicehub.net/register} site then after verifying our e-mail
address log in to the site.
After that in the ``Projects" menu(1.) add a new project (2.):

\begin{figure}[H]
    \centering
    \includegraphics[width=0.9\textwidth]{figures/devicehub-addproject.png}
\end{figure}

In the pop-up window provide an arbitrarily chosen name and click on new project. Optionally an image and description
can be provided for the project.

\begin{figure}[H]
    \centering
    \includegraphics[width=0.9\textwidth]{figures/devicehub-newproject.png}
\end{figure}

On the next page we can add our device and get the unique IDs for the project that are going to be needed later on.

\begin{figure}[H]
    \centering
    \includegraphics[width=0.9\textwidth]{figures/devicehub-adddevice.png}
\end{figure}

In the pop-up window provide a name for our device. On this same page we can add additional attributes like image and
description but these are not mandatory.

\begin{figure}[H]
    \centering
    \includegraphics[width=0.9\textwidth]{figures/devicehub-newdevice.png}
\end{figure}

\section{MQTT communication and configuring the Gateway}

The sensor and the actuator communicates within their own networks. 
If data transmission over the Internet is required then a \emph{gateway} is required.
The gateway's responsibility is to the place the sensor network data into the appropriate
MQTT channel. Furthermore it also can translate/transcode the data for matching the
data processor's expected format. In this class the sensor network's data is in raw binary format
for providing a compact representation. However the utilized cloud provider expects and sends
data in JSON format. The gateway has to be configured for translating between the two domains.

Code sample for the gateway configuration: 
\url{https://developer.mbed.org/teams/BME-SmartLab/code/DeviceHubNet_DEMO/}

\subsection{Introduction to NRF gateway configuration}

In this class the sensor and the actuator communicates with the external world through
a NRF24L01+ radio module. This radio module operates in the ISM band but it is not so 
widespread its signal can not be received with basic devices and settings. Therefore a
gateway is required that will translate the signals of NRF24L01+ radio module to IP packets
that can be received by any host/program.

The gateway also serves another purpose. Sensors are heavily resource constrained devices as 
a consequence the resources used for communication must also be kept at a minimal level.
Therefore the format of the radio communication usually has a compact binary format.
The maximum frame size of the NRF24L01+ radio is 32 bytes that also requires compact transmission
format. In contrast a human readable format is used over the Internet since
the constraints are different over that communication medium and it is more preferred that way.
A commonly used format for describing data is JSON. The second task of the gateway will be then
to translate between the raw binary sensor data and the JSON format used by the IoT cloud.

The third responsibility of the gateway is to perform a mapping between the physical and virtual
devices. The physical sensors are addressed by their micro-controller's network address and an 
internal type ID that's valid internally in a micro-controller.
In the virtual domain the devices are identified using different identifiers. The gateway
has to perform an address translation so that the physical and virtual sensors can be paired.

\begin{figure}[H]
    \centering
    \includegraphics[width=0.9\textwidth]{figures/gateway.png}
    \caption{The end-to-end communication path}
    \label{fig:gateway}
\end{figure}

\subsection{Structure of the Gateway}

The gateway is implemented using a Raspberry Pi micro computer that also has an identical NRF24L01+
radio module just as the sensor. The Internet connection of the gateway is over an Ethernet link.
The Raspberry Pi device is not always located in the lab and it gets its IP address dynamically
that will be specified by the lab demonstrator.

The gateway runs a Debian based Linux distribution. The software components necessary for
implementing the gateway functionality has been pre-installed that will be controlled by the
lab demonstrator. However in order to execute the lab exercises it will be necessary for the students
to log in to the Raspberry Pi and configure their devices. The access details will also be provided
by the lab demonstrator.

\subsection{Radio connection translation}

For radio transmission both the gateway and the sensors use a NRF24L01+ based radio module.
During the measurements the gateway radio operates on a given channel. The sensor's radio
channel has to be aligned with the gateway's radio channel. The current radio channel will be
given by the lab demonstrator.

During the translation the gateway converts the data frames arriving from the NRF24L01+ side 
to UDP packets and sends them to a predefined host. As a result of this conversion the
UDP packet will carry the sensor node's network address (2 bytes) and also a microsecond accurate
timestamp (16 bytes) that indicates the time of the frame reception.
Upon receiving an UDP packet the gateway uses the first 2 bytes of the UDP payload as the
network address of the sensor and the remaining part of the payload is sent to the sensor unaltered.
The sensor node will see this as if the gateway with ID 0 has sent a message to it, the IP level source
can not be identified from there.

Another task of the gateway on the radio level is to respond to incoming ``PING" messages. 
The sensors are sending ping like messages for testing the radio channel's availability.
The gateway responds with a ``PONG" message for these messages.

\subsection{Content translation}

The translation of the content of the messages is done by a proxy code running on the gateway.
This program is listening on a localhost UDP port and if it receives a packet it runs the
translation logic and sends the results in MQTT format to \emph{devicehub.net}. This program
also subscribes to all MQTT channels on which it expects messages from the direction of \emph{devicehub.net}.
If a message arrives it executes the translation logic then sends the result over the 
configured UDP port.

The gateway is configured in way that the previously described radio connection translation 
module and the content translator proxy are communicating each other. This implements a complete
gateway functionality as shown on Figure~\ref{fig:gateway}

\subsection{Message content translation}

The IoT cloud used on this measurement -- \emph{devicehub.net} -- discriminates two types of 
data: \emph{ANALOG} and \emph{DIGITAL}. ANALOG type can represents an arbitrary floating point
value while DIGITAL is an on/off type of data that can only have values 0 and 1.
During the communication one message can only carry one value as a consequence on the sensor
network side each data has to be sent in a separate message.

Values of type DIGITAL are translated to 1 byte of payload on the NRF side while the values
of type ANALOG are translated to a 4 byte \emph{float} type. The sensors have to send data
corresponding to these formats.

An example of the translation for the sensor to devicehub direction:


\begin{tabular}{| l | l | l |}
\hline
 & Received data (payload field, without type) & Forwarded data (data field, without channel) \\
 \hline
DIGITAL type & 0x01 & {"value" : 1} \\
\hline
ANALOG type & 0x42 0x2a 0x00 0x00 & {"value" : 42.5} \\
\hline
\end{tabular}


\appendix

\section{Entry quiz sample questions}

\begin{enumerate}
    \item Describe briefly the main concept of the OpenFlow recommendation.
    \item What are the components of an OpenFlow network?
\end{enumerate}

\section{Lab exercises}

1.0 Hello World
Mérési feladat

Jelentkezzen be az mbed oldalára. Válassza ki a platformok közül a NUCLEO-F446RE-t.
Készítsen programot, amely a fejlesztői lapon található ledet villogtatja.
Használja az mbed által nyújtott példaprogramokat!
Készítsen programot, amely a lapon található gomb segítségével vezérli a ledet.
Készítsen programot, amely képes a méréshez kapcsolt PC-vel kommunikálni. A kommunikációhoz soros terminált
használjon. (A "minicom" ajánlott)

2.1 Szenzor és beavatkozó illesztése
Mérési feladat

Válasszon egyet a rendelkezésre álló szenzorok és beavatkozók közül és készítsen/keressen hozzá kódot, amellyel a
szenzor értékeit megjelenítheti a PC-n.
Keresse meg a szenzorhoz tartozó adatlapot és tanulmányozza! Az illesztés során ügyeljen az adatlapon leírtak
betartására!
Illessze a szenzort a laphoz közvetlenül vagy breadboard segítségével. Amennyiben szükséges használjon egyéb
alkatrészeket.
Készítsen programot, amely a szenzor adatait szabályos időközönként megjeleníti tömör formátumban.

2.2 Kommunikáció illesztése
Mérési feladat

Illessze a rádiós egységet a laphoz. Az rádió SPI felületen csatlakozik. Azonosítsa a megfelelő lábakat és kösse
össze azokat. Ügyeljen, hogy a szenzor és a rádió ne zavarja egymás kommunikációját.
A rádió illesztéséhez szükséges programkönyvtárakat töltse le az mbed gyűjteményéből, illetve kérje el a
mérésvezetőtől.
Ismerje meg a kommunikációs modul használatát a mérésvezetőtől kapott példaprogramon keresztül.
Ellenőrizze a kommunikációt az átjárónál. Próbálja ki, hogy oda-vissza működik a kommunikáció.

2.3 Szenzor és beavatkozó életrekeltése
Mérési feladat

Egyesítse a szenzor és a kommunikáció kódját. A szenzor által mért jeleket küldje el az átjárónak.
A meglévő kódhoz adja hozzá a beavatkozó kódját.

3.1 Virtuális eszköz készítése
Mérési feladat

Jelentkezzen be a devicehub.net oldalára és végignavigálva ez egyes menüpontokon készítse el első projektjét, a
projekthez tartozó első virtuális eszközét, majd az ahhoz tartozó szenzort és beavatkozót.
Jegyezze fel a szenzor és a beavatkozó eléréséhez szükséges paramétereket.
Vizsgálja meg miként érhetőek el a szenzor és beavatkozó adatok az MQTT protokollon keresztül.

3.2 MQTT kommunikáció
Mérési feladat

Tanulmányozza át az MQTT protokollt. Vizsgálja meg a megoldás egyes elemeit.
Egy MQTT kapcsolódásra képes eszközzel létesítsen kapcsolatot egy MQTT brokerrel, majd küldjön és fogadjon
üzeneteket a program segítségével. A mérőgépen az MQTTfx nevű alkalmazás van telepítve. Használhat más alkalmazást is.
Küldjön üzeneteket a virtuális eszközének és fogadjon onnan üzeneteket. Az üzenetek formátumát és a csatornákat az
előző pontban megvizsgált leírásban találja.

4.1 Átjáró beállítása
Mérési feladat

Kérje el a mérésvezetőtől az átjáró eléréséhez szükséges paramétereket (Az átjáró elérése mérésenként változhat)
A megtalálható példák segítségével konfigurálja az átjárót, hogy a megfelelő formátumra alakítást végezze, illetve
a megfelelő csatornát válassza.
Küldjön valós szenzoradatokat a virtuális eszköznek és a valós aktuátor fogadjon virtuális adatokat.

\subsection{Lab environment}

\end{document}