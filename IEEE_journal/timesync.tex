% !TeX encoding = UTF-8
%% (requires IEEEtran.cls version 1.7 or later) with an IEEE conference paper.
\documentclass[journal]{IEEEtran}

\renewcommand\IEEEkeywordsname{Keywords}

\pagenumbering{gobble}

\usepackage[utf8]{inputenc}

% *** GRAPHICS RELATED PACKAGES ***
%\usepackage[pdftex]{graphicx}
\usepackage{graphicx}
%\usepackage[dvips]{graphicx}
% to place figures on a fixed position
\usepackage{float}

% *** PDF, URL AND HYPERLINK PACKAGES ***
\usepackage{url}

% correct bad hyphenation here
\hyphenation{NetFPGA}

\usepackage{xcolor}



% \renewcommand\note[1]{} % uncomment this line to hide notes

%%%%%%%%%%%%%%%%%%%%%%%%%%%%%%%%%%%%%%%%%%%%%%%%%%%
%%%%%%%%%%%%%%%%%%%%%%%%%%%%%%%%%%%%%%%%%%%%%%%%%%%
%%%%%%%%%%%%%%%%%%%%%%%%%%%%%%%%%%%%%%%%%%%%%%%%%%%
%


% LaTeX quick ref
%
% \cite{refname} to place citation
%
% \label{label_name} to place a label, which can be reference by \ref{label_name}
%
% new paragraph -> empty line between text
%
% \noindent to not indent paragraphs first line
%
% create list with : \begin{itemize} \end{itemize}
% \begin{itemize
% \renewcommand to renew numbering \labelitemi{--} to select bullet type
% \item item elem 1
% \item item elem2
% \end{itemize}
%
% et alia (et al.) should be emphasized (i.e in italic) with \emph{et al.}
%
% to add figure, htb is placement selector , !overrid internal paramters
%\begin{figure}[!htb]
%    \centering
%    \includegraphics[width=0.5\textwidth]{FIG.png}
%    \caption{Caption}
%    \label{fig:label}
%\end{figure}
%
% ~ concatenates dynamic text with literals
%
% long dash is --
%
% `is single quoted' , ``is double qouted"
%
% to autoformat with latexindent: latexindent -w  -m -l defaultSettings.yaml ProtoImplFPGA.tex


\begin{document}
% conference papers do not typically use \thanks and this command
% is locked out in conference mode. If really needed, such as for
% the acknowledgment of grants, issue a \IEEEoverridecommandlockouts
% after \documentclass
% paper title
% can use linebreaks \\ within to get better formatting as desired
\title{Time synchronization solution for FPGA based distributed Network Monitoring}
%\titleheader{25th Telecommunications forum TELFOR 2017 \hfill Serbia, Belgrade, November 21-22, 2017.}

% author names and affiliations
% use a multiple column layout for up to three different
% affiliations
\author{Ferenc Nándor Janky} %\thanks{Ferenc Nándor Janky	is with the Department of \mbox{Telecommunications} and
%        \mbox{MediaInformatics},
%        Faculty of Electrical Engineering and Informatics, Budapest University of Technology and \mbox{Economics},
%        Magyar tudósok
%        körútja 2., 1117 Budapest, Hungary (\mbox{phone: +36704213213}; \mbox{e-mail:
%            fecjanky@gmail.com})}%
%}%

%\markboth{25th Telecommunications forum TELFOR 2017 \hfill Serbia, Belgrade, November 21-22,
%    2017.}%
%{}

%\IEEEpubid{\makebox[\columnwidth]{978-1-5386-3073-0/17/\$31.00~\copyright~2017 IEEE\hfill}
%    \hspace{\columnsep}\makebox[\columnwidth]{ }}

% make the title area
\maketitle

\begin{abstract}
    \boldmath
    TODO(Pali)
\end{abstract}

\begin{IEEEkeywords}
    FPGA, Networking, Protocol stack, VHDL
\end{IEEEkeywords}

% no keywords
\section{Introduction}\label{sec:Motivation}

TODO(Pali): 1.oldal

\section{Challenges and Requirements in detail}

TODO(Fec) : 0.5 oldal
- Monotonic clock
- Remote locations, skewing clock
- Adaptation to legacy systems 
- Precision requirements (Fendler Tomi TDK)

\section{Related Work}

\subsection{Network Monitoring}
TODO(Pali): 1.oldal


\subsection{FPGA based packet processing}
 TODO(Pali): 1.oldal

\IEEEpubidadjcol

\section{Architecture of the distributed time synchronized monitoring system}

\subsection{Generic concept}
TODO(Fec) : 0.5 oldal
- External network sync, local high precision syn

\subsection{External time synch. subsystem}
TODO(Fec) : 0.5 oldal PTP vs. NTP ( PTP requires network capabilities)

\subsection{Internal time synch. subsystem}
TODO(Fec) : 0.5 oldal
 - HiSTI

\section{Implementation details}

\subsection{External time synch. subsystem}
TODO(Fec) : 0.5 oldal

\subsection{Internal time synch. subsystem}
TODO(Fec) : 0.5 oldal

\section{Verification \& Results}

TODO(Fec) : 1.5 oldal

- Describe the verification method
- Matlab graph, long-term vs. short-term 


\section{Conclusion}

TODO(Fec \& Pali): 0.5 oldal



Refernces: 0.5 oldal

% http://www.ctan.org/tex-archive/biblio/bibtex/contrib/doc/
% The IEEEtran BibTeX style support page is at:
% http://www.michaelshell.org/tex/ieeetran/bibtex/

%\bibliographystyle{IEEEtran}
% argument is your BibTeX string definitions and bibliography database(s)
%\bibliography{references}

%
% <OR> manually copy in the resultant .bbl file
% set second argument of \begin to the number of references
% (used to reserve space for the reference number labels box)

%\begin{thebibliography}{1}
%
%\bibitem{IEEEhowto:kopka}
%H.~Kopka and P.~W. Daly, \emph{A Guide to \LaTeX}, 3rd~ed.\hskip 1em plus
% 0.5em minus 0.4em\relax Harlow, England: Addison-Wesley, 1999.
%
% \end{thebibliography}

\end{document}
