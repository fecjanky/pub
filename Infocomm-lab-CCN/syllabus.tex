\documentclass[a4paper]{article}

\usepackage[utf8]{inputenc}

\usepackage{url}
\usepackage[]{hyperref}

\usepackage{caption}

\usepackage{listings}

\usepackage{color}

% *** GRAPHICS RELATED PACKAGES ***
%\usepackage[pdftex]{graphicx}
\usepackage{graphicx}
%\usepackage[dvips]{graphicx}
% to place figures on a fixed position
\usepackage{float}

\usepackage[margin=1in]{geometry}

\title{Big Data systems and technologies}
\author{}
\date{}


\begin{document}

\maketitle

\tableofcontents

\section{Introduction}

The goal of this laboratory is to give insights into the practical application 
of today's mostly used big data technologies, mainly from the aspect of 
questions concerning networks. The following Wikipedia page contains a 
decent introduction to the concept of big data for beginners: \url{https://en.wikipedia.org/wiki/Big_data}

\section{Data Serialization}

Big data technologies support both "scale up" and "scale out" performance
enhancement for processing large amounts of data. This means that they mostly 
run on multiple PC-s and servers. The processes running on individual 
hosts require to serialize the data before sending it to each other, and
deserialize it after receiving. Further information on the following Wikipedia page:
\url{https://en.wikipedia.org/wiki/Serialization}
\textbf{Being familiar with XML, JSON and YAML formats is mandatory!}


\section{Client-server messaging libraries}

For traditional client-server, point-to-point type communication, there are many
open-source messaging libraries available. In all of the projects, the most 
important aspect is obviously the high transfer performance. The two most
well-known of these are Netty
\url{http://netty.io/} 
and MINA
\url{https://mina.apache.org/}.

\section{Distributed messaging libraries}

Distributed messaging libraries give access to multipoint-to-multipoint type messaging,
and so grant fast data flow between hosts running representatives of a big data technology
in the network. One of the widely used libraries like this is ZeroMQ
\url{http://zeromq.org/}.
Compared to the following message brokers, this paradigm gives access to messaging
without a bottleneck:
\url{http://zeromq.org/whitepapers:brokerless}.

\section{Message brokers}
Message broker applications give a common point between publishers and subscribers (just like a
multiplexing-demultiplexing gateway), and execute the distribution, multiplication, and transmission of
messages. The description of message brokers for beginners can be found on the following
Wikipedia page:
\url{https://en.wikipedia.org/wiki/Message_broker}
\textbf{Being familiar with Kafka},
\url{http://kafka.apache.org/}
\textbf{RabbitMQ},
\url{https://www.rabbitmq.com/}
\textbf{and the basic concepts concerning brokers (queue, topic, key, value) is mandatory!}

\appendix

\section{Entry quiz sample questions}

\begin{enumerate}
\item What does 3V (4V at IBM) mean in the case of big data?
\item Why is investigating network communication important when dealing with big data?
\item What kind of paradigms exist for communication among network entities, and what are some examples?
\item Which are the most well-known data-serializing formats, and why are they necessary?
\item What does an object and a list look like in JSON?
\item Which are the most well-known fast point-to-point network libraries, and in what languages are these implemented?
\item Which is the most well-known multipoint-to-multipoint network library, and what are its main aspects?
\item What are message brokers, how do they work and why are they necessary?
\item What are the main aspects of Kafka?
\item What are the main aspects of RabbitMQ?
\end{enumerate}

\section{Lab exercises}

\subsection{Tasks}

\begin{enumerate}
\item Install Kafka, RabbitMQ, Wireshark and ZeroMQ!
\item Write a script that generates JSON's with adjustable size in a controllable series, and publishes them to a given topic of a given type of broker on a given port of a given host!
\item Write a script that subscribes to a given topic of a given type of broker on a given port of a given host, and records the time of receiving messages!
\item Measure the speed of publishing to the broker, and the speed of a message arriving from an other publisher fellow student!
\item Summarize the results measured both in Kafka and in RabbitMQ in a table or diagram!
\item Send the same message-series to the other fellow student with ZeroMQ, and record performance! Repeat with more receiving colleagues!
\item Observe (with Wireshark) and illustrate the network traffic generated by the previous steps!
\end{enumerate}

\end{document}