\documentclass[a4paper]{article}

\usepackage[utf8]{inputenc}

\usepackage{url}
\usepackage[hidelinks]{hyperref}

\usepackage{caption}

\usepackage{listings}

\usepackage{color}

% *** GRAPHICS RELATED PACKAGES ***
%\usepackage[pdftex]{graphicx}
\usepackage{graphicx}
%\usepackage[dvips]{graphicx}
% to place figures on a fixed position
\usepackage{float}

\usepackage[margin=1in]{geometry}

\title{Big Data systems and technologies - syllabus for demonstrators}
\author{}
\date{}


\begin{document}

\maketitle

\tableofcontents

\section{Tips}

\begin{itemize}
\item Kafka: the host name has to be updated in \textbf{server.properties} file for a working communication
\item Kafka: Start JVM with extra options: \textbf{export \_JAVA\_OPTIONS="­Djava.net.preferIPv4Stack=true" }
\item RabbitMQ: An IPv4 listening address had to be specified in file \textbf{/etc/rabbitmq/rabbitmq.config} instead of an IPv6
\item RabbitMQ: Remote login without authorization is not allowed. Configure user/pass `guest/guest' for remote login
\end{itemize}

\section{Message generation script}
%
\begin{lstlisting}[language=bash,breaklines=true]
#!/bin/bash
#parameters:
# paramters:
# $1 - host
# $2 - count
# $3 - size
# $4 - port
# $5 - topic
for i in $(seq 1 1 $2)
do
echo "{MSG:\"$(cat /dev/urandom | tr -dc 'a-zA-Z0-9' | fold -w $3 | head -n 1)\"}" | /home/user/kafka/bin/kafka-console-producer.sh --broker-list $1:$4 --topic $5 > /dev/null
done;

\end{lstlisting}

\section{Message receiver script}

\begin{lstlisting}[language=bash,breaklines=true]
#!/bin/bash
#parameters:
# paramters:
# $1 - host
# $2 - port
# $3 - topic

/home/user/kafka/bin/kafka-console-consumer.sh --zookeper $1:$2 --topic $3 | \
awk '{print strftime("%Y-%m-%d%r") , $0 }'

\end{lstlisting}
%

\section{Responsible person for this lab, and previous demonstrator}
Dr. Toka László 

\section{Previous experiences}
Usually the students reach the point after installation of the required softwares and writing the publisher/consumer scripts. There is no time for testing the transmission/network/many-to-many functionalities.

\section{TODOs}
\begin{itemize}
\item prepare an image with the installed softwares, and potentially hand out the scripts for them.
\end{itemize}

\end{document}