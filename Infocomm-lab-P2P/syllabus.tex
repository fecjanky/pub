\documentclass[a4paper]{article}

\usepackage[utf8]{inputenc}

\usepackage{url}
\usepackage[hidelinks]{hyperref}

\usepackage{caption}

\usepackage{listings}

\usepackage{color}

% *** GRAPHICS RELATED PACKAGES ***
%\usepackage[pdftex]{graphicx}
\usepackage{graphicx}
%\usepackage[dvips]{graphicx}
% to place figures on a fixed position
\usepackage{float}

\usepackage[margin=1in]{geometry}

\title{P2P – syllabus}
\author{}
\date{}


\begin{document}

\maketitle

\tableofcontents

\section{Introduction}

Compared to traditional client-server connections peer-to-peer solutions (P2P) offer more direct and dynamic alternatives. They provide searching services in a global network and allow direct, point to point data transfers bypassing the administrative and performance bottlenecks of servers.
The well-know and most used P2P applications -- like file exchange, VoIP and on-line gaming -- affects many users and generate the significant portion of today's Internet traffic. Covering all aspects of P2P network during this lab would be impossible therefore only some key parts will be demonstrated. Since file transfer is currently to most utilized application we are going to cover that during this lab.

\section{OpenFlow recommendation}


\appendix

\section{Entry quiz sample questions}

\begin{enumerate}
    \item Describe briefly the main concept of the OpenFlow recommendation.
\end{enumerate}

\section{Lab exercises}

\subsection{Lab environment}



\end{document}