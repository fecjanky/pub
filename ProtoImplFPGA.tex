% !TeX encoding = UTF-8
%% (requires IEEEtran.cls version 1.7 or later) with an IEEE conference paper.
\documentclass[conference]{IEEEtran}


% *** GRAPHICS RELATED PACKAGES ***
%\usepackage[pdftex]{graphicx}
\usepackage{graphicx}
%\usepackage[dvips]{graphicx}
% to place figures on a fixed position
\usepackage{float}

% *** PDF, URL AND HYPERLINK PACKAGES ***
\usepackage{url}

% correct bad hyphenation here
%\hyphenation{}

\usepackage{xcolor}


\newcommand\note[1]{\textcolor{red}{#1}}
% \renewcommand\note[1]{} % uncomment this line to hide notes

%%%%%%%%%%%%%%%%%%%%%%%%%%%%%%%%%%%%%%%%%%%%%%%%%%%
%%%%%%%%%%%%%%%%%%%%%%%%%%%%%%%%%%%%%%%%%%%%%%%%%%%
%%%%%%%%%%%%%%%%%%%%%%%%%%%%%%%%%%%%%%%%%%%%%%%%%%%
%




\begin{document}


% conference papers do not typically use \thanks and this command
% is locked out in conference mode. If really needed, such as for
% the acknowledgment of grants, issue a \IEEEoverridecommandlockouts
% after \documentclass
% paper title
% can use linebreaks \\ within to get better formatting as desired
\title{Protocol Implementations using FPGAs}
% author names and affiliations
% use a multiple column layout for up to three different
% affiliations
\author{\IEEEauthorblockN{Ferenc Nandor Janky}
\IEEEauthorblockA{AITIA International Inc.\\
Telecommunication Division\\Budapest, Hungary\\
Email: \{pvarga, lkovacs, tothfalusi\}@aitia.ai}
}


% make the title area
\maketitle


\begin{abstract}
\boldmath
this is the abstract
\end{abstract}


% no keywords

% LaTeX quick ref
%
% \cite{refname} to place citation
%
% \label{label_name} to place a label, which can be reference by \ref{label_name}
%
% new paragraph -> empty line between text
%
% \noindent to not indent paragraphs first line
%
% create list with : \begin{itemize} \end{itemize}
% \begin{itemize
% \renewcommand to renew numbering \labelitemi{--} to select bullet type
% \item item elem 1
% \item item elem2
% \end{itemize}
%
% et alia (et al.) should be emphasized (i.e in italic) with \emph{et al.}
%
% to add figure, htb is placement selector , !overrid internal paramters
%\begin{figure}[!htb]
%    \centering
%    \includegraphics[width=9cm]{FIG.png}
%    \caption{Caption}
%    \label{fig:label}
%\end{figure}
%
% ~ concatenates dynamic text with literals
%
% long dash is --
%
% `is single quoted' , ``is double qouted"


\section{Introduction}

some introdcution

\section{Related Work}\label{sec:RelatedWork}

here I should introduce some related works

\section{Conclusion}

And summarize with a nice conclusion

\section{Acknowledgement}
We would like to thank to all of our colleagues and students who contributed to the success of our project


% can use a bibliography generated by BibTeX as a .bbl file
% BibTeX documentation can be easily obtained at:
% http://www.ctan.org/tex-archive/biblio/bibtex/contrib/doc/
% The IEEEtran BibTeX style support page is at:
% http://www.michaelshell.org/tex/ieeetran/bibtex/

\bibliographystyle{IEEEtran}
% argument is your BibTeX string definitions and bibliography database(s)
\bibliography{references}

%
% <OR> manually copy in the resultant .bbl file
% set second argument of \begin to the number of references
% (used to reserve space for the reference number labels box)

%\begin{thebibliography}{1}
%
%\bibitem{IEEEhowto:kopka}
%H.~Kopka and P.~W. Daly, \emph{A Guide to \LaTeX}, 3rd~ed.\hskip 1em plus
% 0.5em minus 0.4em\relax Harlow, England: Addison-Wesley, 1999.
%
% \end{thebibliography}


\end{document}


