\documentclass{article}

\usepackage[utf8]{inputenc}

\usepackage{url}

\usepackage{caption}

% *** GRAPHICS RELATED PACKAGES ***
%\usepackage[pdftex]{graphicx}
\usepackage{graphicx}
%\usepackage[dvips]{graphicx}
% to place figures on a fixed position
\usepackage{float}

\title{OpenFlow \& Mininet – syllabus}
\author{}
\date{}


\begin{document}

\maketitle

\section{Introduction}

Today's networking devices vendors hide implementation details for trivial reasons, hence the researchers have a serious problem testing new networking protocols, innovative ideas in real-world networks, under real-world conditions, as there is no way to modify the details of the implementation. Naturally one can build test networks from consumer PCs and running new switching or routing algorithms on these, but they operate on much lower speeds and has far less processing power compared to carrier-grade switches, routers, which solve most tasks from hardware. This problem is solved by the OpenFlow recommendation, which separates the internal operation of the switches and control logic, enabling the option of programmable networks (SDN - Software Defined Networking). Thus, simple hardware-based basic operations remain very fast and manufacturers do not have to disclose their details while the controller is separate and programmable.

Before this lab, it is advisable to review the following lectures on Network Construction and Operation:

\section{OpenFlow recommendation}

OpenFlow is actually a system that provides a unified interface/protocol for modifying the behaviour of switches, so they do not transmit packets according to the ``normal" mode of operation, but as determined by the programmer. This description aims to briefly present the OpenFlow framework. For more details, please see the following documents:
\begin{itemize}
\item Open Networking Foundation (OpenFlow Standardization Organization) \url{http://www.opennetworking.org/}
\item Newer version of OpenFlow whitepaper: Software-Defined Networking: The New Norm for Networks \url{https://www.opennetworking.org/sdn-resources/sdn-library/whitepapers}
\end{itemize}

The project also aims to be interoperable with the existing networking infrastructure, meaning that traditional traffic continues to processed by the networking devices while using OpenFlow to experiment with new packet forwarding procedures. Accordingly, OpenFlow can be implemented in hardware switches (major manufacturers have already done this in their products).OpenFlow is a relatively new technology that was released in 2008 at Stanford University. However, as more and more companies started to be interested in this technology and started to use it more widely, the system's shortcomings came to light that had been addressed by subsequent releases of the recommendation. As a result of the contribution from major manufacturers and suppliers, the standardization activity grew beyond the academic environment and a separate organization had been created (Open Networking Foundation). Currently Version 1.5 is the latest OpenFlow recommendation, but in most cases the devices support version 1.0 only. There are, of course, some exceptions, and there are now tools available that can work according to standard version 1.3.

It is important to note that currently version 1.0 is the most frequently used release however a significant change -- that affects the architecture of the hardware implementations -- was published between versions 1.0 and 1.1. Therefore we will deal with these two versions in detail in this syllabus.

As from version 1.1, hardware variants are not that simple to be implemented (if you want to utilize the existing elements and experiences), and nowadays software computing potential you can implement very powerful switch implementations in plain software environments, significant improvements have also been made in this direction. Under a software switch, we mean a solution that can run on a computer -- such as a desktop PC -- (in kernel or user space) and can manage the network cards in the machine as separate switch ports. 

Such software tools include the OpenFlow reference switch implemented at Stanford University, which operates in user-space (which is, of course, less efficient); or Open vSwitch, which is a much more complex virtual switch in kernel space that can be used not only for OpenFlow, but is used for example in an operating system virtualization hosting software as a virtual network interface. (For Xen, this is the default virtual switch, but it is also supported by VirtualBox.) The reference switches can also be ported to the OpenWRT environment, so it is possible to create OpenFlow-capable switches from low-cost devices that run OpenWRT. Furthermore there are more expensive hardware devices that are able to operate  according to the OpenFlow recommendation after the proper firmware upgrade. For example, HP ProCurve 6600 that is also available in the lab inventory. For the sake of completeness, it should be noted that OpenFlow was implemented on a NetFPGA card, where operation is implemented with re-programmable hardware devices (FPGAs).

\subsection{OpenFlow system architecture}

OpenFlow's concept is based on the fact that most manufacturers' NPUs (Network Processing Unit) contains one or more forwarding table - a.k.a. flow table - that controls the packet forwarding process. The OpenFlow standard separates the devices into two categories as shown in Figure~\ref{fig:OpenFlow-arch}. The switch performs packet forwarding based on its flow table, while the entries in the table are populated by an external controller. This allows switches to be "dumb" devices, so there is no need for serious computing performance to run routing protocols etc. , as this is done by a separate entity -- the controller.


 An additional advantage is that the specific implementation can be proprietary as long as it is operating according to the OpenFlow protocol and implementing the specification as the "programming" of the system is mainly performed via the external controller, which is an open system. 

 It is important to mention that OpenFlow's packet transfer components are called switches, but these devices are capable of much more compared to conventional L2 switches. 
 The L2 switches are capable of forwarding packets based on L2 headers. In some cases, this may be associated with several extra functions (e.g. STP - Spanning Tree Protocol), but normally L3 headers are never inspected by a traditional L2 switch. 

 In contrast, the OpenFlow switches are able to interpret the L2 and L3 header fields as described in more detail below. This means that an OpenFlow switch can also have L3 routing implemented, i.e it can also function as an IP router. Accordingly, the switch term refers to OpenFlow switches that are capable of packet forwarding both on L2 and L3 (even L4) header fields.

\begin{figure}[!htb]
    \centering
    \includegraphics[width=0.9\textwidth]{figures/OpenFlow-architektura-2.jpg}
    \caption{OpenFlow architecture}
    \label{fig:OpenFlow-arch}
\end{figure}

 The controller uses the OpenFlow protocol to communicate with the switches -- optionally via a secure channel. In OpenFlow 1.1, it has appeared as a novelty in comparison to 1.0 that it has multiple flow tables, and there are not just flow table, but also group tables which can be used for packet replication effectively - e.g IP multicast. Multiple flow tables can be useful because one can group different types of process entries, for example, in the first table ACL (Access Control List) information can be stored, while the second can be used for MPLS (MultiProtocol Label Switching) entries and so on. The structure of the entries in the process tables is shown in Table~\ref{tab:flow-entries}:

\begin{table}
\centering
 \begin{tabular}[b] {| l | l  | l |} 
 \hline
 Header fields & Counters & Actions \\
 \hline
 \end{tabular}
 \caption{Flow table entry structure}
 \label{tab:flow-entries}
\end{table}


The header fields are those parts of the packet headers that are used by the switch to make the forwarding decision by  finding a matching flow entry in the flow table. The fields that are used as the flow key for the lookup are show in Table~\ref{tab:flow-key}

\begin{table}
\centering
 \begin{tabular}[b] {| l |} 
 \hline
 Ingress port \\ \hline 
 Meta-data \\ \hline 
 Source MAC address \\ \hline 
 Dest. MAC address \\ \hline 
 Ether type \\ \hline 
 VLAN id \\ \hline 
 VLAN priority \\ \hline 
 MPLS label \\ \hline 
 MPLS class of service \\ \hline 
 IPv4 source address \\ \hline 
 IPv4 destination address \\ \hline 
 IP Protocol \\ \hline 
 IPv4 ToS \\ \hline 
 Source TCP/UDP port \\ \hline 
 destination TCP/UDP port \\ 
 \hline
 \end{tabular}
 \caption{Flow key structure}
 \label{tab:flow-key}
\end{table}

All but the meta-data fields are well specified. Meta-data is used by OpenFlow to exchange information between tables, as the packet can go through several tables during processing, searching for a matching entry, but a match in one table may affect the search in a subsequent table. The meta-data field is suitable for transmitting these kind of information and it is stored in 64 bits.

Packages arriving to the OpenFlow switch go into a pipeline where forwarding decision will be made. This is illustrated on Figure~\ref{fig:OpenFlow-pipeline}.

\begin{figure}[!htb]
    \centering
    \includegraphics[width=0.9\textwidth]{figures/table-pipeline.png}
    \caption{Packet processing pipeline}
    \label{fig:OpenFlow-pipeline}
\end{figure}

The incoming packets go through the tables one after another until a match is found in any of the entries. If the package matches an entry, the counters of that entry are updated and the related instructions/actions are executed. Also as shown in Figure~\ref{fig:OpenFlow-pipeline}, between the pipeline stages the packet, the associated meta-data and the action set to be performed are being forwarded conceptionally. 

When the packet reaches the end of the last flow table or there is an instruction in the matching entry to execute the actions, the actions in the action set will be executed. After that, the packet either leaves the switch on the corresponding port (s) or it is discarded. Figure~\ref{fig:OpenFlow-matching-process} augments this process with the possibility of a jump between tables and shows what happens when a matching entry is not found in a table.
The \emph{`Drop'} terminal on Figure~\ref{fig:OpenFlow-matching-process} -- based on the table configuration -- can be either :

\begin{itemize}
	\item send to controller
	\item drop
	\item continue with next table
\end{itemize}

\begin{figure}[!htb]
    \centering
    \includegraphics[width=0.9\textwidth]{figures/openflow-processing-flowchart.png}
    \caption{Flow entry matching process}
    \label{fig:OpenFlow-matching-process}
\end{figure}

Figure~\ref{fig:OpenFlow-matching-process} also emphasizes the role of the controller: if the switch doesn't find a matching entry during the flow table lookup it potentially embeds the packet in an OpenFlow packet and it sends that to the OpenFlow controller using the OpenFlow protocol. The controller can then act accordingly -- it can simply ignore the event, or alternatively it can install new  or modify existing flow entries, or it can inject control packets into the network.

The flow entries can contain the following instructions:
\begin{itemize}
\item \emph{`Execute action set'} : execute some immediate action(s) without affecting the action set of the packet
\item \emph{`Delete action set'} : this will empty the action set
\item \emph{`Write meta-data'} : write some appropriately masked meta-data into the packet associated meta-data register
\item \emph{`Goto Table'} : specifies that in which subsequent table should the switch continue the lookup procedure. The table ID must be greater than the current table id.
\end{itemize}

The action set can contain the following actions:
\begin{itemize}
	\item Send out packet on a specific port. The port can be physical or logical like for example the \emph{Controller port} or \emph{All} which can be used to send out packets on all ports
	\item Drop packet
	\item Send to group table
	\item Modify various packet header fields
\end{itemize}

\subsubsection{OpenFlow 1.0 simplifications, differences}

In OpenFlow 1.0 specification there is only one flow table instance, so there is no pipeline, and there are actions directly instead of instructions because as there is no need to collect the actions in each table. In reality there are more than one table present in the v1.0 implementations. A typical software implementation has two of them -- the first one is a hash table, in which the so-called exact match entries are stored. These entries have all the header fields data populated which are specified in the OpenFlow specification used for matching. The second is used for storing the so-called \emph{`wild-card'} entries. Typical implementation is that the wild-card entry is accompanied with a bit mask that is applied to the packet data used in the matching process. As a result some fields can be excluded from the packet forwarding decision.  For example, for HP hardware devices used in this lab, also have a third flow table with hardware acceleration. (But these tables are not operating in a pipeline!)

\subsection{The OpenFlow channel}

The OpenFlow enabled switches connect to the controller via this interface. The channel can be established over TLS (Transport Layer Security) encrypted or unencrypted TCP connection. Communication over the OpenFlow channel is done through the OpenFlow protocol. The protocol distinguishes three message types, these are the router-switch, asynchronous, and symmetric messages. 

Controller-switch messages are initiated by the controller and the switch may not always respond to them. For example, configuring the switch or querying its capabilities falls into this category but statistics queries or flow table modifications are also belong to this category. 

In contrast, asynchronous messages are sent by the switch to the controller in an unsolicited manner. These messages for example can indicate changes in  operational state change of switch ports (up / down), other errors, removal of flow entries due to time-out, or if there is no matching entry for a packet that is transmitted to the controller in this way. 

Symmetric messages can be initiated by either party unsolicited but the originator expects a reply for these types of messages. This includes the \emph{Hello} message used for channel establishment, and the \emph{Echo} message that is used for keep-alive mechanism and measuring the bandwidth of the channel.

\section{Controllers - NOX controller}

The most important element of an OpenFlow-based network is the controller that controls the switches. It is essential that the controller and the controller-switch connection are functioning properly for the normal operation of the network. There are several suggestions for implementing redundancy, such as a distributed controller or redundant connections. 
However, if the controller-switch is still down, two operating modes are possible according to the specification. One is \emph{``fail secure mode"}. In this mode the switch will continue to operate using the existing entries (but they still removed if a time-out is configured for them) and it drops the packets destined to the controller -- a.k.a \emph{``headless mode"}. This mode of operation provides resiliency in case of short interruption of the switch-controller channel.

Another mode is the "fail standalone mode". In this mode, the switch will return to traditional mode, i.e. without using OpenFlow, it will operate as a conventional Ethernet switch. This mode can also be operational in case of a longer controller failure, but the network may not work according to expectations. The preferred mode of operation in case of controller outage can be configured freely in the switch.

Today, a number of control platforms are available to use, where network applications can be developed using different software environments and programming languages/paradigms. One of the earliest, oldest controllers is called NOX. The NOX Controller was basically designed for the most efficient operation in mind, currently this platform -- implemented in C++ -- provides the best performance in term of the number of streams served under a period of time (for a conventional desktop PCs -- which, of course, can mean anything -- that means more than 10,000 new streams per second). The trade-off for efficiency and relatively good scalability is the not so convenient programming environment  using C++ language. The goal of the creators of NOX is to create a network operating system (see NOX: Towards an Operating System for Networks). In this context Networking Operating System means that the software is not intended to manage the network, but it provides a well-defined programming interface for the network itself that makes it is easier to implement the centralized control of the networking equipments from different vendors and to write portable and re-usable control programs that implement some networking functionality. They have created an analogy with traditional operating systems, as they are hiding the details of the underlying hardware and have well-defined programming interfaces that allows the developers to write portable, higher-level programs. NOX can be programmed in python as well, but there is a pure python-based controller is available called POX. That is implemented entirely in python, while in the classical version the core is implemented in C++ and python bindings are provided to be able use it from python.

 POX provides a much clearer, simpler programming environment, especially useful for rapid prototyping and educational purposes. One disadvantage (perhaps) is that it currently only supports version 1.0 of OpenFlow. Several other controllers are available, such as Floodlight (Java based), OpenDaylight (also a Java based but enterprise grade), Ryu (python based), Trema (ruby or C language programmable) or higher abstraction such as Frenetic (declaratively programmable), ONOS (a network operating system).

The NOX controller is also available for OpenFlow versions 1.1, 1.2, 1.3 and 1.4 which is used for demonstration in the following examples. NOX has a modular architecture and it can be extended with custom made applications where existing functionalities (e.g. discovery modules) can be re-used as well. The applications can be implemented using C ++ or Python.

NOX implements the central control using the OpenFlow protocol, as it's shown on Figure~\ref{fig:NOX-network}.
The system itself runs on one or more central servers. The OpenFlow switches connect to the servers through which the network can be controlled. The server also runs additional applications beside the controller (typically one more) that uses the NOX API to manage the network. The network view -- as seen by NOX -- is stored in a database called \emph{`Network View'}. The view is created dynamically by querying information from each OpenFlow switch.

\begin{figure}[!htb]
    \centering
    \includegraphics[width=0.9\textwidth]{figures/NOX_Networks.png}
    \caption{The components of a NOX based network}
    \label{fig:NOX-network}
\end{figure}


\subsection{Demonstration of operation on a simple topology}

In this section a simple OpenFlow based topology is presented on which the operating principles can be demonstrated in practice. The topology is illustrated by Figure~\ref{fig:NOX-sample-topo} (and it is actually emulated using the Mininet framework presented in Section~\ref{sec:Mininet})

\begin{figure}[!htb]
    \centering
    \includegraphics[width=0.9\textwidth]{figures/sample_topology.jpg}
    \caption{Sample topology}
    \label{fig:NOX-sample-topo}
\end{figure}

The topology consists of two switches and two host attached to each of it -- the identifiers used can be read from Figure~\ref{fig:NOX-sample-topo}.
The switches are connected to a NOX based controller on which a typical MAC address learning application is running. Querying the flow entries from the switches makes it clear that all flow tables are empty after start-up (see Figure~\ref{fig:NOX-empty-flowtable}) similarly to a conventional L2 switch with an empty MAC table that is populated dynamically as source MAC addresses are observed of incoming packets. Despite that the switches have been discovered by the controller furthermore the connected devices are also visible (network view, the controller is topology aware)

\begin{figure}[!htb]
    \centering
    \includegraphics[width=0.9\textwidth]{figures/empty_flowtable.png}
    \caption{Empty flow table after start-up}
    \label{fig:NOX-empty-flowtable}
\end{figure}

After initialization a ping is executed from host \emph{h3} to \emph{h4} that results in a bidirectional flow establishment in a way that corresponding flow entries are populated into the flow table's of the switches. The flow entries after the ping are shown by Figure~\ref{fig:NOX-populated-flowtable}.

\begin{figure}[!htb]
    \centering
    \includegraphics[width=0.9\textwidth]{figures/populated_flowtable.png}
    \caption{Populated flow table}
    \label{fig:NOX-populated-flowtable}
\end{figure}

As a result of inspecting the ping messages timing it is clear that the forwarding time of the first packet takes considerably longer time (2.33 ms) compared to the subsequent packets (0.06-0.2 ms) because the flow entries are populated as the controller reacts to the OpenFlow messages triggered by the first packet events as illustrated by Figure~\ref{fig:NOX-ping}

\begin{figure}[!htb]
    \centering
    \includegraphics[width=0.9\textwidth]{figures/ping_example.png}
    \caption{Ping messages and their timings}
    \label{fig:NOX-ping}
\end{figure}

The flow tables provisioning is done by sending the corresponding flow mod messages. Figure~\ref{fig:NOX-populated-flowtable} shows that two entries have been added in table 0. -- as expected -- , that represent bi-directional forwarding with the corresponding dl\_src and dl\_dst MAC addresses set furthermore the IP addresses have been provisioned also. These entries do not contain wild-cards so the controller was able to fill our each mandatory field in the flow key. In contrast to this a conventional L2 switch only forwards based on the destination MAC address and floods the packet if the destination address is unknown.

Such wild-carded entries can be created in the OpenFlow framework so that the masked out parts of the flow key do not participate in the matching process. It can be also observed that the entries have a 5 seconds inactivity time-out -- after which they are removed -- plus the Instructions to be carried out are also there.

The OpenFlow control messages can be monitored using Wireshark packet capture software over the control interface. Currently only OpenFlow v1.0 has an adequate protocol dissector plug-in.

\subsection{OpenFlow v1.0 differences and simplifications}

\begin{figure}[!htb]
    \centering
    \includegraphics[width=0.9\textwidth]{figures/populated_flowtable.png}
    \caption{Populated flow table}
    \label{fig:NOX-populated-flowtable}
\end{figure}

\begin{figure}[!htb]
    \centering
    \includegraphics[width=0.9\textwidth]{figures/populated_flowtable.png}
    \caption{Populated flow table}
    \label{fig:NOX-populated-flowtable}
\end{figure}

\end{document}